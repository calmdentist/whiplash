\documentclass[11pt, a4paper]{article}

% --- UNIVERSAL PREAMBLE BLOCK ---
% Set geometry for sensible margins
\usepackage[a4paper, top=2.5cm, bottom=2.5cm, left=2cm, right=2cm]{geometry}

% Load font and language packages
\usepackage{fontspec}
\usepackage[english, bidi=basic, provide=*]{babel}

% Provide the main language (English)
\babelprovide[import, onchar=ids fonts]{english}

% Set default/Latin font to Sans Serif in the main (rm) slot
% Noto Sans is clean and modern, good for a whitepaper.
\babelfont{rm}{Noto Sans}
% --- END UNIVERSAL PREAMBLE BLOCK ---

% --- Additional Packages ---
\usepackage{amsmath} % For math notation like P(A)
\usepackage{booktabs} % For professional tables
\usepackage{graphicx} % For placeholder box
\usepackage{titling} % For author/title formatting
\usepackage{parskip} % Removes paragraph indentation and adds space between them

% --- hidelinks MUST be the last package loaded ---
\usepackage[hidelinks]{hyperref}

\title{\textbf{Whiplash: A Novel AMM for Unified Spot and Leverage Trading}}
\author{
    calmxbt\\
    \href{https://github.com/calmdentist/whiplash}{https://github.com/calmdentist/whiplash}
}
\date{June 2025}

\begin{document}

\maketitle

\begin{abstract}
This whitepaper introduces \emph{Whiplash}, a novel automated market maker (AMM) that allows both spot and leverage trading within a unified liquidity framework for long-tail assets. By employing the Uniswap V2 style invariant with modifications to accommodate leveraged positions, Whiplash enables a rich trading experience while ensuring the underlying AMM remains solvent at all times. Whiplash also allows for asset creation without seed capital for assets that meet certain criteria. We describe the mathematical foundation - operational mechanics for opening and closing leveraged positions and liquidations.
\end{abstract}

\section{Introduction}

Existing perpetual DEXs are based on the vAMM model, which resembles traditional perpetual futures contracts built on an AMM. However, this model has several limitations:

\begin{itemize}
    \item Fragmented liquidity: separate liquidity pool for leverage trading.
    \item Counterparty risk: Longs require shorts as counterparties to trade, and vice-versa.
    \item Oracles: Oracles are used to fetch spot prices from external sources. This assumes trust and creates attack vectors.
\end{itemize}

Whiplash unifies spot and leverage trading within a single liquidity pool - providing deeper liquidity, reducing counterparty risk, and eliminating the need for oracles.

\section{Mathematical Foundation}

At its core, Whiplash uses a constant product invariant similar to Uniswap V2. However, to support unified spot and leverage trading, the protocol introduces the following concepts: \textbf{borrowed liquidity} (represented by $\Delta k$) and the \textbf{LP funding rate}. 

\\\\When leverage traders open positions, they effectively borrow liquidity from the pool, which temporarily reduces the liquidity depth. This borrowed liquidity is repaid continuously through the LP funding rate and settled when positions close.

\\\\The LP funding rate increases quadratically as the pool becomes increasingly leveraged, incentivizing a healthy equilibrium.

\subsection{Core Variables}

The protocol tracks the following state variables:
\begin{itemize}
    \item $x$: The total reserve of the base asset (real + virtual).
    \item $y$: The total reserve of the token (real + virtual).
    \item $x_l$: The leveraged base asset amount allocated to short positions.
    \item $y_l$: The leveraged token amount allocated to long positions.
    \item $k_{\text{original}}$: The original constant product at pool launch.
    \item $\Delta k_i$: The borrowed liquidity for position $i$ (stored with each position).
    \item $\sum \Delta k_i$: The total borrowed liquidity across all open positions.
\end{itemize}

The fundamental constant product invariant is:
\begin{equation}
x \cdot y = k
\end{equation}

When positions are open, the effective constant product is reduced by the total borrowed liquidity:
\begin{equation}
k_{\text{effective}} = k_{\text{current}} - \sum \Delta k_i
\end{equation}

\section{Leverage Trading Mechanism}

Similarly to traditional derivative contracts, leveraged positions on Whiplash are virtual claims on pool value rather than physical token ownership.

\subsection{Opening a Leveraged Position}

When a trader opens a leveraged position with collateral $c$ and leverage multiplier $L$, the position size is calculated as though the trader swapped the notional amount ($c$ x $L$) against the constant product AMM:

\begin{equation}
\Delta y = \frac{y_{\text{pre}} \cdot c \cdot L}{x_{\text{pre}} + c \cdot L}
\end{equation}

This $\Delta y$ becomes the position's virtual size, $y_{\text{position}}$. Critically, these tokens are not physically transferred out of the pool - they remain in the vault. The position is a pure accounting entry representing a claim on pool value.

\\\\The borrowed liquidity, $\Delta k$, is calculated as the reduction in the constant product caused by opening the position:

\begin{equation}
\Delta k = k_{\text{pre}} - k_{\text{post}} = (x_{\text{pre}} \cdot y_{\text{pre}}) - ((x_{\text{pre}} + c) \cdot (y_{\text{pre}} - \Delta y))
\end{equation}

This $\Delta k$ value is stored with the position and represents the liquidity borrowed from the pool. The trader must repay this borrowed liquidity through:
\begin{enumerate}
    \item \textbf{Continuous funding payments}: Accrued over the position's lifetime at a rate that increases quadratically with the pool's total leverage
    \item \textbf{Final settlement}: Any remaining debt is settled when closing the position
\end{enumerate}

\section{Global LP Funding Rate}
To address the temporary reduction in effective liquidity caused by open leveraged positions, Whiplash implements a global LP funding rate. This mechanism requires leverage traders to continuously pay funding fees to the liquidity pool, compensating spot traders for the borrowed liquidity. The funding rate is designed to increase quadratically with total leverage, creating a strong economic incentive to maintain a healthy pool state.

\subsection{Funding Rate Formula}
The funding rate is a function of the total borrowed liquidity relative to the pool's original constant product. Define the leverage ratio:
\begin{equation}
    r = \frac{\sum \Delta k_i}{k_{\text{original}}}
\end{equation}

The funding rate per unit time is then:
\begin{equation}
    \text{FundingRate} = C \cdot \left( \frac{r}{1 - r} \right)^2 = C \cdot \left( \frac{\sum \Delta k_i}{k_{\text{original}} - \sum \Delta k_i} \right)^2
\end{equation}

where $C$ is a configurable constant (default: 0.0001 per second). The second form is computationally optimized and equivalent to the first.
\\\\
This function has critical properties:
\begin{itemize}
    \item When $r = 0$ (no leverage), funding rate is zero
    \item As $r \to 1$ (total $\Delta k$ approaches original $k$), funding rate $\to \infty$
    \item Quadratic growth ensures leverage becomes prohibitively expensive before systemic risk
    \item Creates a natural economic cap on total system leverage
\end{itemize}

\subsection{Implementation via Cumulative Index}
The protocol uses two global accumulators to efficiently track funding without iterating over positions:
\begin{itemize}
    \item $\text{unrealized\_fees}$: Total fees owed to the pool, updated as $\Delta \text{unrealized\_fees} = \text{FundingRate}(r) \cdot (\sum \Delta k_i) \cdot \Delta t$
    \item $I(t)$: Cumulative funding rate index, updated as $I(t_{\text{new}}) = I(t_{\text{old}}) + \text{FundingRate}(r) \cdot \Delta t$
\end{itemize}
When a position opens, the current $I(t)$ is stored with the position for later fee calculation.

\section{Closing a Leveraged Position}

When closing, the protocol calculates funding fees accrued, computes the effective $\Delta k$, determines the payout, and transfers directly from pool to user:

\begin{enumerate}
    \item Calculate funding fees: $\text{funding\_due} = (I(t_{\text{close}}) - I(t_{\text{open}})) \cdot \Delta k_{\text{original}}$
    \item Convert to $\Delta k$ repayment: $\Delta k_{\text{repaid}} = \text{funding\_due} \cdot y_{\text{current}}$
    \item Compute effective debt: $\Delta k_{\text{effective}} = \Delta k_{\text{original}} - \Delta k_{\text{repaid}}$
    \item Calculate payout: $X_{\text{out}} = \frac{x_{\text{current}} \cdot y_{\text{position}} - \Delta k_{\text{effective}}}{y_{\text{current}} + y_{\text{position}}}$
\end{enumerate}

The payout already incorporates funding fees. Upon closing, the position's $\Delta k_{\text{original}}$ is removed from $\sum \Delta k_i$, funding fees are realized, and the funding rate is updated.

\section{Liquidation Mechanism}

A position becomes liquidatable when $X_{\text{out}} \leq \frac{\Delta k_{\text{effective}}}{x_{\text{current}}} \times 1.05$ (5\% buffer). The liquidator receives a reward equal to $y_{\text{position}} - \Delta y_{\text{restore}}$ where $\Delta y_{\text{restore}} = \frac{\Delta k_{\text{effective}}}{x_{\text{current}}}$ is the amount needed to restore the pool's invariant. Since positions are virtual, liquidation simply updates pool accounting and transfers the reward directly from the vault to the liquidator.

\section{Limbo State}

A position enters "limbo" when it meets the liquidation condition but no liquidator has closed it (unprofitable). It can exit limbo if price movements cause recovery: $x_{\text{current}} \cdot y_{\text{position}} > \Delta k_{\text{effective}}$. This protects traders from flash crash liquidations. Underwater positions still pay the LP funding rate, and thus will go to 0 (and fully repay their debt) over time.

\section{Liquidity Provisioning and Launching}

Whiplash enables permissionless token launches with zero seed capital, using a bonding curve model similar to pump.fun. Traders who buy the token on the bonding curve essentially fund the LP once the token reaches the bonding threshold and graduates to the AMM.

\section{System Properties}

Whiplash's design provides several important properties:

\subsection{Solvency}

The protocol is always solvent because the pricing mechanism ensures that the output can never exceed the real reserve, regardless of input size.

\subsection{No Seed Capital Requirement}

The bonding curve based token launch mechanism allows early traders to become liquidity providers, and eliminates the need for external LPs.

\subsection{Economic Incentive Alignment}

The quadratic funding rate and virtual position model create powerful incentive alignment across all participants:
\begin{itemize}
    \item \textbf{Leverage traders}: Early closure incentivized when pool leverage is high; continuous debt repayment through funding
    \item \textbf{Spot traders}: Real-time benefit from LP funding rate: $k_{\text{effective}} = k_{\text{current}} + \text{unrealized\_fees} \cdot y_{\text{current}}$ in swap pricing, providing better execution as if funding fees were already realized
    \item \textbf{Protocol}: Natural economic cap on total leverage ($\sum \Delta k_i \to k_{\text{original}}$ makes funding $\to \infty$)
\end{itemize}

\section{Considerations and Future Work}

Whiplash makes the tradeoff of temporarily reducing effective liquidity for traders in order to power leverage, which is restored through the LP funding rate and settled when the positions are closed/liquidated. This tradeoff was chosen because in order to not reduce effective liquidity for traders, solvency would not be guaranteed in extreme cases (bank runs).
\\\\
However, if liquidations were integrated at the protocol level (i.e. swaps liquidate positions that would become underwater), effective liquidity could remain constant. This is an ideal scenario - designing such a liquidation mechanism that can stay within the limits of compute usage in smart contracts is very difficult, if possible.
\\\\
Future work will involve designing a liquidation engine integrated at the protocol level, and exploring other fundamentally different AMM mechanisms (concentrated liquidity/liquidity bins for example) that create a net improvement to the protocol.

\end{document} 