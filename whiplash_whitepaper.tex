\documentclass[11pt]{article}
\usepackage[utf8]{inputenc}
\usepackage{amsmath,amssymb,amsthm}
\usepackage{algorithm}
\usepackage{algorithmic}
\usepackage{listings}
\usepackage{graphicx}
\usepackage{hyperref}
\usepackage{geometry}
\usepackage{tikz}
\usepackage{xcolor}
\geometry{letterpaper, margin=1in}

\definecolor{whiplashpurple}{RGB}{128, 0, 128}

\hypersetup{
    colorlinks=true,
    linkcolor=whiplashpurple,
    filecolor=whiplashpurple,
    urlcolor=whiplashpurple,
    citecolor=whiplashpurple
}

\title{\textbf{Whiplash: A Novel AMM for Unified Spot and Leverage Trading}}
\author{calmxbt}
\date{\today}

\begin{document}

\maketitle

\begin{abstract}
This whitepaper introduces \emph{Whiplash}, a novel automated market maker (AMM) designed specifically for the memecoin market. Whiplash makes an already volatile asset class even more volatile by combining spot and leverage trading within a unified liquidity framework, requiring zero seed capital for new token launches. By employing the Uniswap V2 style invariant with modifications to accommodate leveraged positions, Whiplash enables a novel trading experience while ensuring the underlying AMM remains solvent at all times. We describe the mathematical foundation, operational mechanics for launching tokens, opening and closing leveraged positions, and a robust liquidation mechanism that protects the protocol during periods of extreme volatility.
\end{abstract}

\section{Introduction}

The memecoin market represents one of the most volatile sectors in the cryptocurrency space. Whiplash aims to amplify this volatility by providing a permissionless platform where traders can engage in both spot and leveraged trading from day zero, without requiring seed capital for liquidity provision. This whitepaper outlines how Whiplash modifies the traditional constant product AMM to facilitate leveraged trading while maintaining protocol solvency.

\subsection{Motivation}
Traditional AMMs, such as Uniswap, require significant seed capital to establish liquidity for new tokens. Additionally, leveraged trading typically exists in separate protocols disconnected from spot markets. Whiplash addresses both limitations by:
\begin{itemize}
    \item Enabling permissionless token creation without seed capital requirements
    \item Unifying spot and leverage trading within a single liquidity pool
    \item Ensuring protocol solvency through a novel approach to token distribution and leverage mechanics
\end{itemize}

\section{Mathematical Foundation}

At the core of Whiplash is a modified Uniswap V2 style AMM with the constant product invariant:

\begin{equation}
x \cdot y = k
\end{equation}

where:
\begin{itemize}
    \item $x$ represents the reserve of the base asset (stablecoin or SOL)
    \item $y$ represents the reserve of the memecoin token
    \item $k$ is a constant value maintained during spot trading
\end{itemize}

Unlike traditional AMMs, Whiplash introduces modifications to accommodate leveraged trading and zero seed capital token launches.

\subsection{Virtual Reserves Model}

For new token launches, Whiplash employs a "virtual reserves" model for the base asset side of the pool:

\begin{equation}
x_{\text{virtual}} \cdot y_{\text{real}} = k_{\text{initial}}
\end{equation}

where:
\begin{itemize}
    \item $x_{\text{virtual}}$ is the virtual reserve of the base asset
    \item $y_{\text{real}}$ is the real reserve containing 100\% of the token supply
    \item $k_{\text{initial}}$ is the initial constant product value
\end{itemize}

This approach enables token creation without requiring real base asset liquidity, as the token supply is fixed and 100\% contained within the liquidity pool at launch.

\subsection{Symmetric Virtual Reserves for Leveraged Positions}

To support leverage in a fully solvent and spread-free manner Whiplash keeps a \emph{pair} of running tallies for open interest:
\begin{align}
L &\;=\; \text{total memecoins borrowed by long positions},\\
S &\;=\; \text{total base assets borrowed by short positions}.
\end{align}

These quantities are treated as \textbf{virtual reserves}.  For spot pricing we therefore use the \emph{effective} balances
\begin{equation}
\tilde x \;=\; x_{\text{real}} + S, \qquad
\tilde y \;=\; y_{\text{real}} + L ,
\quad\text{so that}\quad
\tilde x \tilde y = k.
\end{equation}

Opening and closing positions simply move balances between the real and virtual books while leaving~$k$ unchanged:
\begin{itemize}
    \item \textbf{Long open} (collateral $c$, receives $\Delta y$ tokens): \\[-1.2ex]
    $x_{\text{real}}\!\leftarrow x_{\text{real}} + c,\;
     y_{\text{real}}\!\leftarrow y_{\text{real}} - \Delta y,\;
     L\!\leftarrow L + \Delta y$.
    \item \textbf{Long close}: reverse the above transfers and pay
    \[ X_{\text{out}} = \tilde x_{\text{before}} - \frac{k}{\tilde y_{\text{before}}+\Delta y}. \]
    \item \textbf{Short legs} are symmetric with $x\leftrightarrow y$ and $L\leftrightarrow S$.
\end{itemize}

Because both sides are exposed symmetrically the pool never promises more of either asset than it controls, and no artificial bid/ask spread is introduced. Setting a weighting factor $0<\gamma\le 1$ in the effective balances (e.g., $\tilde x = x_{\text{real}} + \gamma S$) offers a simple dial to charge an internal ``funding'' spread when desired.

\section{Token Launch Mechanism}

The token launch process in Whiplash represents a significant innovation in the AMM space:

\subsection{Initial State}
When a new memecoin token is created:

\begin{equation}
\begin{aligned}
y_{\text{total}} &= \text{Total token supply} \\
y_{\text{pool}} &= y_{\text{total}} \quad \text{(100\% of tokens in pool)} \\
x_{\text{virtual}} &= \text{Initial virtual base asset reserve}
\end{aligned}
\end{equation}

The initial constant product is established as:

\begin{equation}
k_{\text{initial}} = x_{\text{virtual}} \cdot y_{\text{pool}}
\end{equation}

\subsection{Zero Sum Game Property}

A key mathematical property that enables Whiplash's innovative approach is the zero-sum nature of the token ecosystem:

\begin{equation}
\forall t: y_{\text{pool},t} + y_{\text{users},t} = y_{\text{total}}
\end{equation}

where $t$ represents any point in time after launch. This invariant ensures that the sum of tokens in the pool and in user wallets always equals the total supply, creating a closed system.

This property guarantees that the virtual reserves model remains solvent because there can never be a scenario where more tokens are demanded from the pool than exist in total circulation.

\section{Leverage Trading Mechanism}

Whiplash introduces a symmetric \emph{virtual reserve} approach to leverage trading that \emph{preserves} the effective constant-product invariant even while positions are open. All open interest is mirrored into virtual reserves that are included in the pricing curve, eliminating insolvency paths and artificial bid/ask spread.

\subsection{Opening a Leveraged Position}

A trader supplies collateral $c$ (in the base asset) and chooses a leverage factor $L\ge 1$.  The operation is executed in two legs while \emph{always} keeping the effective product $k$ untouched.

\paragraph{1.~Spot leg.}  Using the effective reserves $(\tilde x,\tilde y)$ the trader swaps the collateral for tokens just like a normal AMM trade:
\begin{align}
\tilde x' &= \tilde x + c,\\
\Delta y_{\text{spot}} &= \tilde y - \frac{k}{\tilde x'} .
\end{align}
Real vault updates: $x_{\text{real}}\!\leftarrow x_{\text{real}}+c$ and $y_{\text{real}}\!\leftarrow y_{\text{real}}-\Delta y_{\text{spot}}$.

\paragraph{2.~Borrow leg.}  To reach the desired leverage the trader borrows extra tokens
\[
\Delta y_{\text{borrow}} = (L-1)\,\Delta y_{\text{spot}} ,
\]
removing them from $y_{\text{real}}$ and adding the same amount to the virtual tally $L$.  Because the real and virtual movements cancel out in $\tilde y$, the effective reserves remain $(\tilde x',\tilde y)$ and therefore $k$ is still exactly preserved.

The position record stores $(c,\,\Delta y_{\text{borrow}})$; its mark-to-market value will evolve with the pool price.

\subsection{Closing a Leveraged Position}

Let the position hold the debt $\Delta y_{\text{borrow}}$ and the purchased tokens $\Delta y_{\text{spot}}$.

\paragraph{1.~Debt repayment.}  The trader first transfers $\Delta y_{\text{borrow}}$ back to the pool.  The real vault gains those tokens and the virtual long tally decreases by the same amount, so $\tilde y$ and $k$ stay unchanged and no price impact occurs.

\paragraph{2.~Spot unwind.}  The remaining $\Delta y_{\text{spot}}$ tokens are now swapped for the base asset using the standard constant-product formula on the current effective reserves $(\tilde x,\tilde y)$:
\[
X_{\text{out}} = \tilde x - \frac{k}{\tilde y + \Delta y_{\text{spot}}} .
\]
The pool pays $X_{\text{out}}$ from $x_{\text{real}}$, after which $(\tilde x- X_{\text{out}})(\tilde y+\Delta y_{\text{spot}})=k$ holds by construction.

\paragraph{PnL.}  The trader's profit or loss on the position is simply
\[
\text{PnL} = X_{\text{out}} - c .
\]

The same logic with $x\leftrightarrow y$ applies to closing shorts.

\section{Liquidation Mechanism}

Because the pool never borrows from itself, insolvency can only arise if a position's collateral no longer suffices to repurchase the debt it owes.  For a long this happens when
\[
X_{\text{debt}} = \tilde x - \frac{k}{\tilde y + \Delta y_{\text{borrow}}} \;\ge c .
\]
If the inequality is met the position is \emph{liquidatable}.  A liquidator simply transfers the missing base $X_{\text{debt}}-c$ to the pool (or equivalently buys the debt tokens) and receives the entire collateral plus an optional bounty.  Shorts are symmetric with $(x\leftrightarrow y)$.

This mechanism closes underwater positions exactly at the breakeven point, guaranteeing that the pool never pays out more than it holds and therefore eliminating bad debt.

\section{System Properties and Guarantees}

Whiplash's design provides several important mathematical guarantees:

\subsection{Solvency Guarantee}

The solvency of the protocol is guaranteed by the zero-sum nature of the token supply and the constant product invariant:

\begin{equation}
\forall t: y_{\text{pool},t} + y_{\text{users},t} + \sum_{i} y_{\text{position},i,t} = y_{\text{total}}
\end{equation}

where $y_{\text{position},i,t}$ represents the tokens in the $i$-th leveraged position at time $t$.

\subsection{No Seed Capital Requirement}

The virtual reserves model eliminates the need for seed capital by ensuring:

\begin{equation}
\forall t: x_{\text{required},t} \leq \sum_{j} \text{deposits}_j
\end{equation}

where $\text{deposits}_j$ represents the base asset deposited by the $j$-th user for either spot or leverage trading.

\subsection{Zero Bad Debt Guarantee}

The protocol guarantees zero bad debt through its liquidation mechanism:

\begin{equation}
\forall \text{ positions } i: \text{either } \begin{cases}
\text{position is healthy} \\
\text{position is liquidatable}
\end{cases}
\end{equation}

Under no circumstances can a position create a debt that the protocol cannot recover, as the fixed token supply ensures all positions are backed by real tokens.

\section{Conclusion}

Whiplash represents a paradigm shift in AMM design by enabling permissionless token launches with zero seed capital and unifying spot and leverage trading within a single framework. The mathematical foundation ensures that:

\begin{enumerate}
    \item New tokens can be launched without requiring liquidity providers
    \item Leverage trading can amplify the inherent volatility of memecoins
    \item The protocol remains solvent at all times through its novel approach to reserve management
\end{enumerate}

By building on proven AMM mechanics while introducing innovative modifications, Whiplash creates a platform that makes "the world's most volatile asset class more volatile," providing a truly novel trading experience for memecoin enthusiasts.

\end{document} 