\documentclass[11pt]{article}
\usepackage[utf8]{inputenc}
\usepackage{amsmath,amssymb,amsthm}
\usepackage{algorithm}
\usepackage{algorithmic}
\usepackage{listings}
\usepackage{graphicx}
\usepackage{hyperref}
\usepackage{geometry}
\usepackage{tikz}
\usepackage{xcolor}
\geometry{letterpaper, margin=1in}

\definecolor{whiplashpurple}{RGB}{128, 0, 128}

\hypersetup{
    colorlinks=true,
    linkcolor=whiplashpurple,
    filecolor=whiplashpurple,
    urlcolor=whiplashpurple,
    citecolor=whiplashpurple
}

\title{\textbf{Whiplash: A Novel AMM for Unified Spot and Leverage Trading}}
\author{
    calmxbt\\
    \href{https://github.com/calmdentist/whiplash}{https://github.com/calmdentist/whiplash}
}
\date{June 2025}

\begin{document}

\maketitle

\begin{abstract}
This whitepaper introduces \emph{Whiplash}, a novel automated market maker (AMM) that allows both spot and leverage trading within a unified liquidity framework for long-tail assets. By employing the Uniswap V2 style invariant with modifications to accommodate leveraged positions, Whiplash enables a rich trading experience while ensuring the underlying AMM remains solvent at all times. Whiplash also allows for asset creation without seed capital for assets that meet certain criteria. We describe the mathematical foundation - operational mechanics for opening and closing leveraged positions and liquidations.
\end{abstract}

\section{Introduction}

Crypto has shown great interest in long-tail assets such as memecoins. However, there is no avenue to trade these assets with leverage. Current perpetual DEXs only allow whitelisted assets with high market capitalizations. Whiplash aims to solve this with a permissionless protocol where traders can engage in both spot and leveraged trading from day zero. This whitepaper outlines how Whiplash modifies the traditional constant product AMM to facilitate leveraged trading while maintaining protocol solvency.

\subsection{Motivation}
Traditional AMMs, such as Uniswap, require significant seed capital to establish liquidity for new tokens. Additionally, leveraged trading typically exists in separate protocols disconnected from spot markets. Whiplash addresses both limitations by:
\begin{itemize}
    \item Enabling permissionless token creation without seed capital requirements
    \item Unifying spot and leverage trading within a single liquidity pool
    \item Ensuring protocol solvency through a novel approach to token distribution and leverage mechanics
\end{itemize}

\section{Mathematical Foundation}

At its core, Whiplash uses a constant product invariant similar to Uniswap V2. However, to support unified spot and leverage trading, the protocol maintains separate accounting for spot reserves and total leveraged debt. The total reserves are composed of:
\begin{itemize}
    \item $x$: The total reserve of the base asset (real + virtual).
    \item $y$: The total reserve of the token.
    \item $x_l$: The total borrowed amount of the base asset from short positions (leveraged SOL).
    \item $y_l$: The total borrowed amount of the token from long positions (leveraged tokens).
\end{itemize}

The fundamental constant product invariant, used for pricing leveraged trades and calculating position values, is:
\begin{equation}
x \cdot y = k
\end{equation}

where $k$ is a constant value. Spot trades, however, use a modified version of this invariant to ensure protocol solvency, as detailed in Section 2.3.

\subsection{Dynamic Reserve Adjustment for Spot Trading}

To maintain solvency without penalizing small spot trades, Whiplash introduces a dynamic adjustment mechanism. When the pool has open leveraged positions, the reserves used for calculating spot trades are adjusted based on the size of the trade. This prevents scenarios where large spot trades could drain the real reserves of the pool faster than the pricing formula anticipates.

The core pricing formulas are modified with a dynamic factor, $\alpha \in [0, 1]$:
\begin{align}
\text{For buys (X to Y): } (x + \alpha \cdot x_l) \cdot y &= k \\
\text{For sells (Y to X): } x \cdot (y + \alpha \cdot y_l) &= k
\end{align}

where:
\begin{itemize}
    \item $x$ and $y$ are the total reserves of the base asset and token, respectively.
    \item $x_l$ is the total borrowed base asset from short positions.
    \item $y_l$ is the total borrowed token from long positions.
    \item $\alpha$ is a dynamic factor that scales with the trade size.
\end{itemize}

The factor $\alpha$ is determined using quadratic interpolation. It is designed to be near zero for small trades, minimizing price impact for average users. As the trade size increases, $\alpha$ approaches 1, applying the full leveraged debt to the calculation. The function for $\alpha$ is defined such that $\alpha=1$ precisely when the trade's output would equal the pool's entire real reserve of the output asset.
\\\\
This approach creates a "soft" boundary that ensures the pool's solvency under extreme conditions while preserving favorable pricing for the majority of spot trades.

\section{Leverage Trading Mechanism}

Whiplash introduces a novel approach to leverage trading that modifies the constant product invariant temporarily while positions are open. To ensure protocol solvency, the change in the invariant is tracked and used to settle the position.

\subsection{Opening a Leveraged Position}

When a trader opens a leveraged position with collateral $c$ and leverage factor $L$, the trade is executed as a spot trade of size $c \cdot L$. The resulting tokens, $\Delta y$, are stored in a position manager.
\begin{equation}
\Delta y = \frac{y_{\text{pre}} \cdot c \cdot L}{x_{\text{pre}} + c \cdot L}
\end{equation}
These $\Delta y$ tokens become the size of the user's position, $y_{\text{position}}$.
\\\\
To track the pool's total debt, the borrowed portion of this trade is calculated. This is the difference between the tokens received from the full leveraged trade and the tokens that would have been received if only the collateral $c$ was swapped:
\begin{equation}
\Delta y_l = \Delta y - \frac{y_{\text{pre}} \cdot c}{x_{\text{pre}} + c}
\end{equation}
This borrowed amount is added to the pool's total leveraged token amount, $y_l$. A similar calculation applies for short positions, increasing $x_l$.
\\\\
Opening the position alters the pool's reserves and its constant product, $k$. The user's collateral $c$ is added to the base asset reserve, and the pool's total leveraged debt is updated.

\begin{equation}
\begin{aligned}
x_{\text{post}} &= x_{\text{pre}} + c \\
y_{\text{post}} &= y_{\text{pre}} - \Delta y \\
y_{l, \text{post}} &= y_{l, \text{pre}} + \Delta y_l \\
k_{\text{post}} &= x_{\text{post}} \cdot y_{\text{post}}
\end{aligned}
\end{equation}

The deviation in the constant product, $\Delta k$, is calculated and stored with the position. This value is crucial for correctly closing the position later.

\begin{equation}
\Delta k = k_{\text{pre}} - k_{\text{post}} = (x_{\text{pre}} \cdot y_{\text{pre}}) - ((x_{\text{pre}} + c) \cdot (y_{\text{pre}} - \Delta y))
\end{equation}

\subsection{Mathematical Impact on Reserves}

For a long position, the reserves are modified as shown above. The stored $\Delta k$ represents the exact amount by which the pool's invariant must be adjusted when the position is closed to ensure the pool remains solvent and the user's profit or loss is correctly calculated.

\section{Closing a Leveraged Position}

When a trader closes a leveraged position, the protocol uses the stored $\Delta k$ to restore the invariant and calculate the final payout. This process ensures that the trader's PnL reflects market movements while keeping the AMM's core mathematics sound.

\subsection{Position Closure Calculation}

To close a position, the user returns their $y_{\text{position}}$ tokens. The protocol calculates the base asset payout, $X_{\text{out}}$, by determining how many base assets must be paid out to restore the original constant product invariant, $k$. The payout is given by:

\begin{equation}
X_{\text{out}} = \frac{x_{\text{current}} \cdot y_{\text{position}} - \Delta k}{y_{\text{current}} + y_{\text{position}}}
\end{equation}

Here, $x_{\text{current}}$ and $y_{\text{current}}$ are the pool's reserves at the time of closing. The term $x_{\text{current}} \cdot y_{\text{position}}$ represents the value of the position's tokens in terms of the base asset, adjusted for swap impact, from which the stored $\Delta k$ is subtracted to ensure the pool's invariant is restored.

This payout, $X_{\text{out}}$, includes the user's initial collateral plus any profits, or minus any losses. The user's Profit and Loss (PnL) on their initial collateral is:

\begin{equation}
\text{PnL} = X_{\text{out}} - c
\end{equation}

Upon closing, the borrowed amount originally associated with the position ($\Delta y_l$ or a corresponding $\Delta x_l$) is subtracted from the pool's total leveraged debt ($y_l$ or $x_l$), ensuring the system's accounting for total debt remains accurate.

\section{Liquidation Mechanism}

The liquidation mechanism incentivizes liquidators with a reward for closing underwater positions while restoring the pool's invariant. This approach ensures efficient liquidation of risky positions while providing clear economic incentives for liquidators.

\subsection{Liquidation Condition}

A position becomes \textbf{liquidatable} when the expected payout from closing the position is insufficient to restore the stored $\Delta k$ with a 5\% buffer. For long positions, the condition is:

\begin{equation}
X_{\text{out}} \leq \frac{\Delta k}{x_{\text{current}}} \times 1.05
\end{equation}

where $X_{\text{out}}$ is the payout the position would receive from a normal close, calculated as:

\begin{equation}
X_{\text{out}} = \frac{x_{\text{current}} \cdot y_{\text{position}} - \Delta k}{y_{\text{current}} + y_{\text{position}}}
\end{equation}

For short positions, the equivalent condition applies with the roles of base asset and token reversed.

\subsection{Liquidation Execution}

When a liquidator calls the \texttt{liquidate} instruction, the protocol executes the following steps:

\begin{enumerate}
    \item The position is closed by returning all $y_{\text{position}}$ tokens to the pool
    \item The exact quantity needed to restore the invariant is calculated:
    \begin{equation}
    \Delta y_{\text{restore}} = \frac{\Delta k}{x_{\text{current}}}
    \end{equation}
    \item This amount is transferred to the pool reserves to restore $k$ exactly
    \item Any remaining tokens from the position become the liquidator's reward:
    \begin{equation}
    \text{Liquidator Reward} = y_{\text{position}} - \Delta y_{\text{restore}}
    \end{equation}
\end{enumerate}

This mechanism ensures the pool's solvency is maintained while providing economic incentives for timely liquidation of underwater positions. The 5\% buffer in the liquidation condition helps account for small price movements and ensures liquidators are adequately compensated for their service.

\section{Limbo State}

A unique feature of Whiplash is the "limbo state" that protects traders from liquidation as a result of flash crashes.

\subsection{Mathematical Definition of Limbo}

A position enters limbo when it meets the liquidation condition but no liquidator has yet closed it (likely because it is unprofitable to do so).
\begin{equation}
\begin{aligned}
&x_{\text{current}} \cdot y_{\text{position}} \le \Delta k \\
&\text{and no liquidator has fulfilled the liquidation}
\end{aligned}
\end{equation}

\subsection{Exiting Limbo}

A position can exit limbo if price movements cause its value to recover, such that it no longer meets the liquidation condition:

\begin{equation}
x_{\text{current}} \cdot y_{\text{position}} > \Delta k
\end{equation}

\section{Token Launch Mechanism}

The token launch mechanism requires no seed capital and allows for arbitrary depth/liquidity. It is designed only for tokens with a fixed supply that launch with 100\% of the supply in the LP.
\\\\The innovation is based on the fact that a token that launches with the above conditions can never go below the launch price (i.e. the base reserve can never go below the initial base reserve). Thus, the initial base reserve may as well be 'virtual' in such cases.
\\\\I'm not going to prove this, but it can be proven using the path invariant property of constant product AMMs.
        

\subsection{Initial State}
When a new memecoin token is created:

\begin{equation}
\begin{aligned}
y_{\text{total}} &= \text{Fixed total token supply} \\
y_{\text{pool}} &= y_{\text{total}} \quad \text{(100\% of tokens in pool)} \\
x_{\text{virtual}} &= \text{Initial virtual base asset reserve}
\end{aligned}
\end{equation}

The initial constant product is established as:

\begin{equation}
k_{\text{initial}} = x_{\text{virtual}} \cdot y_{\text{pool}}
\end{equation}

\section{System Properties and Guarantees}

Whiplash's design provides several important mathematical guarantees:

\subsection{Solvency Guarantee}

The solvency of the protocol is guaranteed because the pricing mechanism ensures that the output can never exceed the real reserve, regardless of input size.

\subsection{No Seed Capital Requirement}

The token launch mechanism assumes a fixed supply and 100\% of the supply in the LP, allowing the initial base asset reserve to be virtual.

\subsection{Zero Bad Debt Guarantee}

The protocol guarantees zero bad debt as debt from underwater positions are essentially absorbed by traders. Underwater positions effectively reduce the constant product K, thus reducing effective liquidity for traders until the position is closed or liquidated.

\section{Considerations and Future Work}

Whiplash makes the tradeoff of temporarily reducing effective liquidity for traders in order to power leverage, which is restored when leveraged positions are closed/liquidated. This tradeoff was chosen because in order to not reduce effective liquidity for traders, solvency would not be guaranteed in extreme cases (bank runs).
\\\\
Extreme scenarios can be avoided by setting limits on the following:
\begin{itemize}
    \item Leverage (multiplier) that can be used
    \item $\Delta_k$ when opening a leveraged position
    \item Total $\Delta_k$ for all open positions
\end{itemize}
\\\\
However, if liquidations were integrated at the protocol level (i.e. swaps liquidate positions that would become underwater), solvency would be guaranteed and effective liquidity could remain constant. This is an ideal scenario - designing such a liquidation mechanism that can stay within the limits of compute usage in smart contracts is very difficult, if possible.
\\\\
Future work will involve designing a liquidation engine integrated at the protocol level, and exploring other fundamentally different AMM mechanisms (concentrated liquidity/liquidity bins for example) that create a net improvement to the protocol.

\end{document} 