\documentclass[11pt]{article}
\usepackage[utf8]{inputenc}
\usepackage{amsmath,amssymb,amsthm}
\usepackage{algorithm}
\usepackage{algorithmic}
\usepackage{listings}
\usepackage{graphicx}
\usepackage{hyperref}
\usepackage{geometry}
\usepackage{tikz}
\usepackage{xcolor}
\geometry{letterpaper, margin=1in}

\definecolor{whiplashpurple}{RGB}{128, 0, 128}

\hypersetup{
    colorlinks=true,
    linkcolor=whiplashpurple,
    filecolor=whiplashpurple,
    urlcolor=whiplashpurple,
    citecolor=whiplashpurple
}

\title{\textbf{Whiplash: A Novel AMM for Unified Spot and Leverage Trading}}
\author{calmxbt}
\date{\today}

\begin{document}

\maketitle

\begin{abstract}
This whitepaper introduces \emph{Whiplash}, a novel automated market maker (AMM) designed specifically for the memecoin market. Whiplash makes an already volatile asset class even more volatile by combining spot and leverage trading within a unified liquidity framework, requiring zero seed capital for new token launches. By employing the Uniswap V2 style invariant with modifications to accommodate leveraged positions, Whiplash enables a novel trading experience while ensuring the underlying AMM remains solvent at all times. We describe the mathematical foundation, operational mechanics for launching tokens, opening and closing leveraged positions, and a robust liquidation mechanism that protects the protocol during periods of extreme volatility.
\end{abstract}

\section{Introduction}

The memecoin market represents one of the most volatile sectors in the cryptocurrency space. Whiplash aims to amplify this volatility by providing a permissionless platform where traders can engage in both spot and leveraged trading from day zero, without requiring seed capital for liquidity provision. This whitepaper outlines how Whiplash modifies the traditional constant product AMM to facilitate leveraged trading while maintaining protocol solvency.

\subsection{Motivation}
Traditional AMMs, such as Uniswap, require significant seed capital to establish liquidity for new tokens. Additionally, leveraged trading typically exists in separate protocols disconnected from spot markets. Whiplash addresses both limitations by:
\begin{itemize}
    \item Enabling permissionless token creation without seed capital requirements
    \item Unifying spot and leverage trading within a single liquidity pool
    \item Ensuring protocol solvency through a novel approach to token distribution and leverage mechanics
\end{itemize}

\section{Mathematical Foundation}

At its core, Whiplash uses a constant product invariant similar to Uniswap V2. However, to support unified spot and leverage trading, the protocol maintains separate accounting for spot reserves and total leveraged debt. The total reserves are composed of:
\begin{itemize}
    \item $x$: The total reserve of the base asset (real + virtual).
    \item $y$: The total reserve of the token.
    \item $x_l$: The total borrowed amount of the base asset from short positions (leveraged SOL).
    \item $y_l$: The total borrowed amount of the token from long positions (leveraged tokens).
\end{itemize}

The fundamental constant product invariant, used for pricing leveraged trades and calculating position values, is:
\begin{equation}
x \cdot y = k
\end{equation}

where $k$ is a constant value. Spot trades, however, use a modified version of this invariant to ensure protocol solvency, as detailed in Section 2.3.

\subsection{Virtual Reserves Model}

For new token launches, Whiplash employs a "virtual reserves" model for the base asset side of the pool:

\begin{equation}
x_{\text{virtual}} \cdot y_{\text{real}} = k_{\text{initial}}
\end{equation}

where:
\begin{itemize}
    \item $x_{\text{virtual}}$ is the virtual reserve of the base asset
    \item $y_{\text{real}}$ is the real reserve containing 100\% of the token supply
    \item $k_{\text{initial}}$ is the initial constant product value
\end{itemize}

This approach enables token creation without requiring real base asset liquidity, as the token supply is fixed and 100\% contained within the liquidity pool at launch.

\subsection{Dynamic Reserve Adjustment for Spot Trading}

To maintain solvency without penalizing small spot trades, Whiplash introduces a dynamic adjustment mechanism. When the pool has open leveraged positions, the reserves used for calculating spot trades are adjusted based on the size of the trade. This prevents scenarios where large spot trades could drain the real reserves of the pool faster than the pricing formula anticipates.

The core pricing formulas are modified with a dynamic factor, $\alpha \in [0, 1]$:
\begin{align}
\text{For buys (X to Y): } (x + \alpha \cdot x_l) \cdot y &= k \\
\text{For sells (Y to X): } x \cdot (y + \alpha \cdot y_l) &= k
\end{align}

where:
\begin{itemize}
    \item $x$ and $y$ are the total reserves of the base asset and token, respectively.
    \item $x_l$ is the total borrowed base asset from short positions.
    \item $y_l$ is the total borrowed token from long positions.
    \item $\alpha$ is a dynamic factor that scales with the trade size.
\end{itemize}

The factor $\alpha$ is determined using quadratic interpolation. It is designed to be near zero for small trades, minimizing price impact for average users. As the trade size increases, $\alpha$ approaches 1, applying the full leveraged debt to the calculation. The function for $\alpha$ is defined such that $\alpha=1$ precisely when the trade's output would equal the pool's entire real reserve of the output asset.

This approach creates a "soft" boundary that ensures the pool's solvency under extreme conditions while preserving favorable pricing for the majority of spot trades.

\section{Token Launch Mechanism}

The token launch process in Whiplash represents a significant innovation in the AMM space:

\subsection{Initial State}
When a new memecoin token is created:

\begin{equation}
\begin{aligned}
y_{\text{total}} &= \text{Total token supply} \\
y_{\text{pool}} &= y_{\text{total}} \quad \text{(100\% of tokens in pool)} \\
x_{\text{virtual}} &= \text{Initial virtual base asset reserve}
\end{aligned}
\end{equation}

The initial constant product is established as:

\begin{equation}
k_{\text{initial}} = x_{\text{virtual}} \cdot y_{\text{pool}}
\end{equation}

\subsection{Zero Sum Game Property}

A key mathematical property that enables Whiplash's innovative approach is the zero-sum nature of the token ecosystem:

\begin{equation}
\forall t: y_{\text{pool},t} + y_{\text{users},t} = y_{\text{total}}
\end{equation}

where $t$ represents any point in time after launch. This invariant ensures that the sum of tokens in the pool and in user wallets always equals the total supply, creating a closed system.

This property guarantees that the virtual reserves model remains solvent because there can never be a scenario where more tokens are demanded from the pool than exist in total circulation.

\section{Leverage Trading Mechanism}

Whiplash introduces a novel approach to leverage trading that modifies the constant product invariant temporarily while positions are open. To ensure protocol solvency, the change in the invariant is tracked and used to settle the position.

\subsection{Opening a Leveraged Position}

When a trader opens a leveraged position with collateral $c$ and leverage factor $L$, the trade is executed as a spot trade of size $c \cdot L$. The resulting tokens, $\Delta y$, are stored in a position manager.
\begin{equation}
\Delta y = \frac{y_{\text{pre}} \cdot c \cdot L}{x_{\text{pre}} + c \cdot L}
\end{equation}
These $\Delta y$ tokens become the size of the user's position, $y_{\text{position}}$.

To track the pool's total debt, the borrowed portion of this trade is calculated. This is the difference between the tokens received from the full leveraged trade and the tokens that would have been received if only the collateral $c$ was swapped:
\begin{equation}
\Delta y_l = \Delta y - \frac{y_{\text{pre}} \cdot c}{x_{\text{pre}} + c}
\end{equation}
This borrowed amount is added to the pool's total leveraged token amount, $y_l$. A similar calculation applies for short positions, increasing $x_l$.

Opening the position alters the pool's reserves and its constant product, $k$. The user's collateral $c$ is added to the base asset reserve, and the pool's total leveraged debt is updated.

\begin{equation}
\begin{aligned}
x_{\text{post}} &= x_{\text{pre}} + c \\
y_{\text{post}} &= y_{\text{pre}} - \Delta y \\
y_{l, \text{post}} &= y_{l, \text{pre}} + \Delta y_l \\
k_{\text{post}} &= x_{\text{post}} \cdot y_{\text{post}}
\end{aligned}
\end{equation}

The deviation in the constant product, $\Delta k$, is calculated and stored with the position. This value is crucial for correctly closing the position later.

\begin{equation}
\Delta k = k_{\text{pre}} - k_{\text{post}} = (x_{\text{pre}} \cdot y_{\text{pre}}) - ((x_{\text{pre}} + c) \cdot (y_{\text{pre}} - \Delta y))
\end{equation}

\subsection{Mathematical Impact on Reserves}

For a long position, the reserves are modified as shown above. The stored $\Delta k$ represents the exact amount by which the pool's invariant must be adjusted when the position is closed to ensure the pool remains solvent and the user's profit or loss is correctly calculated.

\section{Closing a Leveraged Position}

When a trader closes a leveraged position, the protocol uses the stored $\Delta k$ to restore the invariant and calculate the final payout. This process ensures that the trader's PnL reflects market movements while keeping the AMM's core mathematics sound.

\subsection{Position Closure Calculation}

To close a position, the user returns their $y_{\text{position}}$ tokens. The protocol calculates the base asset payout, $X_{\text{out}}$, by determining how many base assets must be paid out to restore the original constant product invariant, $k$. The payout is given by:

\begin{equation}
X_{\text{out}} = \frac{x_{\text{current}} \cdot y_{\text{position}} - \Delta k}{y_{\text{current}} + y_{\text{position}}}
\end{equation}

Here, $x_{\text{current}}$ and $y_{\text{current}}$ are the pool's reserves at the time of closing. The term $x_{\text{current}} \cdot y_{\text{position}}$ represents the value of the position's tokens in terms of the base asset, adjusted for swap impact, from which the stored $\Delta k$ is subtracted to ensure the pool's invariant is restored.

This payout, $X_{\text{out}}$, includes the user's initial collateral plus any profits, or minus any losses. The user's Profit and Loss (PnL) on their initial collateral is:

\begin{equation}
\text{PnL} = X_{\text{out}} - c
\end{equation}

Upon closing, the borrowed amount originally associated with the position ($\Delta y_l$ or a corresponding $\Delta x_l$) is subtracted from the pool's total leveraged debt ($y_l$ or $x_l$), ensuring the system's accounting for total debt remains accurate.

\section{Liquidation Mechanism}

The liquidation mechanism protects the protocol by ensuring that underwater positions are closed before they generate bad debt. It relies on the same $\Delta k$ mechanic to settle positions.

\subsection{Liquidation Condition}

A position becomes \textbf{liquidatable} when the payout it would receive from a normal close is \emph{insufficient} to restore the stored $\Delta k$. Mathematically this corresponds to the case where the numerator of Equation~(\ref{eq:payout-long}) (or its short-side analogue) is non-positive so that

\begin{equation}
X_{\text{out}} \le 0 .
\end{equation}

For the long side this reduces to the simple inequality

\begin{equation}
x_{\text{current}}\,y_{\text{position}} \le \Delta k ,
\end{equation}

and the symmetric condition $y_{\text{current}}\,x_{\text{position}} \le \Delta k$ for shorts. At that point the trader's collateral is entirely exhausted and external intervention is required to bring the pool back to solvency.

\subsection{Liquidation Execution}

When a liquidator calls the \texttt{liquidate} instruction the protocol first computes the \emph{exact} quantity needed to mend the invariant:

\begin{equation}
\Delta y_{\text{needed}} = \left\lceil \frac{\Delta k}{x_{\text{current}}} \right\rceil \qquad\text{(longs)},
\end{equation}

or the mirrored expression $\Delta x_{\text{needed}} = \lceil \Delta k / y_{\text{current}} \rceil$ for shorts. Only this amount is transferred from the position vault to the pool; any surplus is left untouched so that the pool's constant product is \emph{exactly} restored—no more, no less. Because the trader's collateral is entirely depleted there is no payout, and the position account is closed with zero balance. The liquidator currently receives no reward, but the mechanism cleanly removes underwater risk from the pool.

\section{Limbo State}

A unique feature of Whiplash is the "limbo state" for positions that experience extreme price movements.

\subsection{Mathematical Definition of Limbo}

A position enters limbo when it meets the liquidation condition but no liquidator has yet closed it:
\begin{equation}
\begin{aligned}
&x_{\text{current}} \cdot y_{\text{position}} \le \Delta k \\
&\text{and no liquidator has fulfilled the liquidation}
\end{aligned}
\end{equation}

\subsection{Exiting Limbo}

A position can exit limbo if price movements cause its value to recover, such that it no longer meets the liquidation condition:

\begin{equation}
x_{\text{current}} \cdot y_{\text{position}} > \Delta k
\end{equation}

\section{System Properties and Guarantees}

Whiplash's design provides several important mathematical guarantees:

\subsection{Solvency Guarantee}

The solvency of the protocol is guaranteed by the zero-sum nature of the token supply and the constant product invariant:

\begin{equation}
\forall t: y_{\text{pool},t} + y_{\text{users},t} + \sum_{i} y_{\text{position},i,t} = y_{\text{total}}
\end{equation}

where $y_{\text{position},i,t}$ represents the tokens in the $i$-th leveraged position at time $t$.

\subsection{No Seed Capital Requirement}

The virtual reserves model eliminates the need for seed capital by ensuring:

\begin{equation}
\forall t: x_{\text{required},t} \leq \sum_{j} \text{deposits}_j
\end{equation}

where $\text{deposits}_j$ represents the base asset deposited by the $j$-th user for either spot or leverage trading.

\subsection{Zero Bad Debt Guarantee}

The protocol guarantees zero bad debt through its liquidation and limbo mechanisms:

\begin{equation}
\forall \text{ positions } i: \text{either } \begin{cases}
\text{position is healthy} \\
\text{position is liquidatable} \\
\text{position is in limbo}
\end{cases}
\end{equation}

Under no circumstances can a position create a debt that the protocol cannot recover, as the fixed token supply ensures all positions are backed by real tokens.

\section{Conclusion}

Whiplash represents a paradigm shift in AMM design by enabling permissionless token launches with zero seed capital and unifying spot and leverage trading within a single framework. The mathematical foundation ensures that:

\begin{enumerate}
    \item New tokens can be launched without requiring liquidity providers
    \item Leverage trading can amplify the inherent volatility of memecoins
    \item The protocol remains solvent at all times through its novel approach to reserve management
    \item Traders are protected from flash crashes through the limbo mechanism
\end{enumerate}

By building on proven AMM mechanics while introducing innovative modifications, Whiplash creates a platform that makes "the world's most volatile asset class more volatile," providing a truly novel trading experience for memecoin enthusiasts.

\end{document} 