\documentclass[11pt]{article}
\usepackage[utf8]{inputenc}
\usepackage{amsmath,amssymb,amsthm}
\usepackage{algorithm}
\usepackage{algorithmic}
\usepackage{listings}
\usepackage{graphicx}
\usepackage{hyperref}
\usepackage{geometry}
\geometry{letterpaper, margin=1in}

\title{Ultrasound: A Novel AMM for Spot and Leverage Trading on the Same Underlying Liquidity}
\author{calmdentist \\ \texttt{calmdentist@proton.me}}
\date{February 26, 2025}

\begin{document}

\maketitle

\begin{abstract}
This whitepaper introduces \emph{Ultrasound}, a novel automated market maker (AMM) that uniquely integrates both spot and leveraged trading within a unified liquidity framework. By employing dynamic virtual reserves together with scaling factors and an integrated progressive liquidation mechanism, Ultrasound enables leveraged positions to adjust effective liquidity while ensuring that the underlying real reserves remain fully solvent. We describe the invariant—including dynamic scaling updates and a novel mathematical model for partial liquidations—the operational logic for opening and closing leveraged positions, and a robust liquidation mechanism that eliminates the risk of bad debt.
\end{abstract}

\section{Introduction}

Automated market makers have revolutionized decentralized trading by allowing liquidity pools to dynamically price assets without traditional order books. However, most AMMs are confined to spot trading, and vAMMs such as Perpetual Protocol fragment liquidity between spot and perpetual pools. \emph{Ultrasound} bridges this gap by offering both spot and leveraged trading on the same underlying liquidity while ensuring risk mitigation through dynamic scaling as well as an integrated liquidation mechanism.

\subsection{Motivation}
Integrating leveraged trading with spot trading in a single liquidity pool enables enhanced capital efficiency and flexible risk management. This whitepaper outlines how virtual reserves, when combined with dynamic scaling factors, adjust the effective liquidity for leveraged trades while safeguarding real reserves against insolvency. A progressive liquidation mechanism is introduced which allows for gradual restoration of the pool invariant, thereby mitigating liquidity gaps in volatile markets.

\section{System Architecture}

At the core of Ultrasound is the concept of combining:
\begin{itemize}
    \item \textbf{Real Reserves:} The actual funds held in the liquidity pool.
    \item \textbf{Virtual Reserves:} Liquidity adjustments introduced for leveraged trading.
\end{itemize}

\subsection{Invariant with Dynamic Scaling}
Unlike traditional invariants, Ultrasound employs scaling factors to ensure that the effective liquidity reflects only the real reserves. The invariant is defined as:
\[
a(x + x_v)\cdot b(y + y_v) = k,
\]
where:
\begin{itemize}
    \item \(x\) and \(y\) denote the real reserves (e.g., ETH and DOG).
    \item \(x_v\) and \(y_v\) denote the virtual reserves added for leveraged trading.
    \item \(a\) and \(b\) are dynamic scaling factors.
    \item \(k\) is a constant that maintains the pool equilibrium.
\end{itemize}

Initially, for an example pool with 1000 ETH and 1000 DOG, we set \(a=b=1\). When leveraged positions are opened, the invariant is preserved while dynamically adjusting \(a\) (or \(b\)) to account only for the real reserve amounts.

\section{Opening a Leveraged Position}

Consider an initial pool with 1000 ETH and 1000 DOG, so that the spot price is \(1\,\text{ETH/DOG}\).

\subsection{Example: Leveraged Long on DOG}
\begin{enumerate}
    \item \textbf{Collateral Deposit:} Bob decides to open a 5x long position on DOG by depositing 10 ETH.
    \item \textbf{Adding Virtual Liquidity:} In addition to Bob's 10 ETH, the protocol adds 40 ETH worth of virtual liquidity for leverage, resulting in an effective ETH reserve of 1050 (with 1010 ETH being real, and 40 ETH virtual). Simultaneously, the swap executed via the invariant delivers approximately 47.62 DOG to Bob, and the DOG reserve becomes 952.38 (all real).
    \item \textbf{Scaling Factor Update:} To ensure that effective liquidity reflects the actual (real) reserves, the scaling factor for ETH is updated so that the effective reserve becomes 1010 ETH rather than 1050. This is achieved by setting
    \[
    a \leftarrow a \times \frac{1010}{1050}.
    \]
    This dynamic adjustment ensures that despite the additional virtual liquidity, the real reserve backing remains unchanged.
\end{enumerate}

\section{Exiting a Position}

For positions that remain sufficiently collateralized, the exit process involves a two-step approach:
\begin{enumerate}
    \item \textbf{Reverting the Scaling Factor:} First, the adjustment is undone so that \(a\) resets via
    \[
    a \leftarrow a \times \frac{1050}{1010}.
    \]
    \item \textbf{Executing the Swap:} Bob's position is then unwound by swapping his 47.62 DOG back to ETH. During this swap, virtual liquidity is deducted along with the premium (if applicable) from both the output and the pool reserve. In our example, the protocol deducts 40 ETH of virtual liquidity, allowing Bob to recover his original 10 ETH collateral.
\end{enumerate}

\section{Liquidation}

A key benefit of the Ultrasound design is that it eliminates the risk of bad debt by ensuring that real reserves always back the effective pool state. However, in volatile market conditions, swaps can push the reserves into ranges that trigger liquidation of open leveraged positions. To avoid abrupt liquidity gaps, Ultrasound employs an integrated, progressive liquidation mechanism.

\subsection{Integrated Liquidation Mechanism}

Rather than triggering all liquidations simultaneously, the mechanism partially liquidates positions to gradually restore the invariant. Let the healthy pool reserves be:
\[
x_0 \cdot y_0 = k_0.
\]
After leveraged trades, if no liquidation occurs, the pool is left in an altered state \((x', y')\) such that:
\[
x' \cdot y' = k < k_0.
\]
Each leveraged position \(i\) (for \(i=1,2,\ldots,N\)) is associated with:
\begin{itemize}
    \item \(d_i\): the virtual DOG liquidity to be restored upon full liquidation.
    \item \(b_i\): the corresponding borrowed ETH (virtual liquidity).
    \item \(y_{\ell,i}\): the liquidation threshold, i.e., the DOG reserve level at which liquidation for that position is triggered.
\end{itemize}
Let the total virtual DOG exposure be:
\[
D = \sum_{i=1}^{N} d_i.
\]

To restore the invariant, we target an effective DOG reserve:
\[
y_{\mathrm{eff}} = y' + \Delta y = \frac{k_0}{x'},
\]
with \(\Delta y\) denoting the aggregate DOG recovered via liquidation. Setting \(\Delta y = \gamma\,D\), the aggregate liquidation fraction is given by:
\[
\gamma = \frac{\frac{k_0}{x'} - y'}{D}.
\]
The following scenarios arise:
\begin{itemize}
    \item \(\gamma=0\): No liquidation is needed.
    \item \(\gamma\geq 1\): All positions are fully liquidated.
    \item \(0 < \gamma < 1\): Only a fraction of the virtual liquidity is liquidated, achieving partial restoration.
\end{itemize}

\subsection{Sequential Liquidation Mechanism}

In practice, positions are ordered by their liquidation thresholds:
\[
y_{\ell,1} \le y_{\ell,2} \le \cdots \le y_{\ell,N}.
\]
Liquidation proceeds sequentially:
\begin{enumerate}
    \item Define the effective DOG reserve after partially or fully liquidating the first \(i\) positions as:
    \[
    y^{(i)} = y' + \sum_{j=1}^{i} \alpha_j\,d_j,
    \]
    where \(\alpha_j \in [0,1]\) represents the liquidation fraction for position \(j\).
    \item For the \(i\)th position, if a partial liquidation is sufficient to raise the effective reserve to its threshold, then choose:
    \[
    \alpha_i = \frac{y_{\ell,i} - \left(y' + \sum_{j=1}^{i-1} \alpha_j\,d_j \right)}{d_i},
    \]
    ensuring \(0\le \alpha_i \le 1\).
    \item Liquidation stops when the effective reserve meets the target:
    \[
    y' + \sum_{j=1}^{m} \alpha_j\,d_j = \frac{k_0}{x'}.
    \]
\end{enumerate}
After liquidation, the adjusted ETH reserve becomes:
\[
x_{\text{new}} = \frac{k_0}{\,y' + \sum_{j=1}^{m} \alpha_j\,d_j},
\]
which guarantees that:
\[
x_{\text{new}} \cdot \left(y' + \sum_{j=1}^{m} \alpha_j\,d_j\right) = k_0.
\]
If even full liquidation of all positions (\(\alpha_i=1\) for all \(i\)) fails to restore the invariant (i.e. if \(\gamma > 1\)), the affected positions enter a \emph{limbo state} because the swap output is insufficient to cover the virtual liquidity.

\subsection{Interpretation and Benefits}

This integrated liquidation mechanism offers several advantages:
\begin{enumerate}
    \item \textbf{Smooth Liquidation:} By liquidating positions partially and in sequence, the protocol avoids sudden jumps, thereby mitigating liquidity gaps.
    \item \textbf{Invariant Restoration:} The sequential recovery ensures that once all positions are liquidated, the pool's reserves return to the invariant \(x_0 \cdot y_0 = k_0\).
    \item \textbf{Capital Efficiency:} Only the necessary fraction of each leveraged position is liquidated based on market conditions, preserving as much real liquidity as possible.
\end{enumerate}

\section{Considerations}
A fundamental feature of Ultrasound's design is that the effective reserves used in swaps may be temporarily augmented by virtual liquidity. However, by dynamically adjusting the scaling factors \(a\) and \(b\), and by applying a progressive liquidation mechanism when necessary, the system ensures that the effective liquidity always reconciles back to the true (real) liquidity. As a consequence, every swap—whether opening/closing a position or executing a liquidation—operates under the principle that:
\[
\text{Real Reserves} \geq \text{Swap Outputs},
\]
thereby eliminating the possibility of bad debt. The tradeoff is that effective liquidity becomes thinner as more leveraged trades occur, but this is corrected once positions are closed or liquidated.

\section{Conclusion and Future Work}
Ultrasound represents a paradigm shift in AMM design by unifying spot and leveraged trading under a single liquidity structure. Key innovations include:
\begin{itemize}
    \item A novel invariant incorporating real and virtual reserves together with dynamic scaling factors.
    \item A dynamic mechanism for restoring pool liquidity through partial, progressive liquidations.
    \item A robust procedure for opening leveraged positions and a resilient liquidation mechanism.
\end{itemize}

Future research will focus on refining scaling factor adjustments further and exploring protocol-level liquidations to ensure absolute risk mitigation while enhancing capital efficiency.

\end{document}